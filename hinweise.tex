\section*{Hinweise zum Schreiben}



\subsection{Grundlegendes}
Zum Thema schriftliche wissenschaftliche Arbeiten empfehle ich folgende Quellen: 

\begin{itemize}
    %\item Auf der Internetseite \href{https://www.f1.htw-berlin.de/studieren/abschlussarbeit/}{\emph{Abschlussarbeiten am FB~1}} befinden sich Antworten zum  formalen Ablauf einer Abschlussarbeit. Fragen wie ''Wie beantrage ich eine Zulassung zur Abschlussarbeit?'' oder ''Kann man die Bearbeitungszeit verlängern?'' werden hier beantwortet. 
    \item Viele Tipps zum Organisieren und Schreiben einer  Abschlussarbeit oder einer wissenschaftlichen Arbeit allgemein findest du im Moodlekurs\\ 
    \href{https://moodle.htw-berlin.de/course/view.php?id=17639}{\emph{Infoportal wissenschaftliches Schreiben}}.
    \item Der offizielle Leitfaden vom FB~1 zum Erstellen einer wissenschaftlichen Arbeit befindet sich hier: \\\href{https://www.f1.htw-berlin.de/fileadmin/HTW/Zentral/FB/FB1/Leitfaden_zur_Erstellung_einer_Wiss.-Arbeit_neu.pdf}{\emph{Leitfaden FB~1}}.\cite{fb1leitfaden}.
    \item In \cite{volkerformeln} wird ausführlich erklärt, wie du Formelzeichen richtig setzt.
\end{itemize}
Dieses Dokument fasst Wesentliches zusammen. 
Außerdem dient dieses Dokument als \LaTeX- Vorlage.

\subsection{Struktur}\label{sec:Struktur}

\begin{itemize}
\item  Auch wenn für die endgültige Fassung kein Inhaltsverzeichnis benötigt wird, macht es Sinn, mit Hilfe des Befehls 
{\tt \textbackslash tableofcontents} während des Schreibens eines zu erstellen. Damit behälst du die Struktur im Auge.
Die Leser sollten schnell wissen, wo sie welche Informationen finden können. Deshalb ist wichtig, klare Überschriften zu verwenden. Keine Abkürzungen in Überschriften.

\item Bemühe dich um Gleichmäßigkeit in Bezug auf die Anzahl der Unterabschnitte pro Abschnitt (Einleitung/ Schluss sind Ausnahmen mit weniger Unterüberschriften). 

\item Zwischen Abschnitts\-über\-schriften und Unter\-ab\-schnitts\-überschriften sollten keine "losen" Textabschnitte eingefügt werden. Ein oder zwei Sätze sind möglich, aber nur für einen Überblick über den gesamten Abschnitt.

\item Ein Abschnitt kann nicht nur einen einzigen Unterabschnitt enthalten.

\item Überspringe keine Überschriftenebenen.

\item Vermeide mehr als drei Ebenen (Abschnitt, Unterabschnitt, Unterunterabschnitt). Wenn du auf einer tieferen Ebene unterteilen müsstest, überarbeite deine Struktur.

\item Jeder Absatz sollte eine eindeutige Idee/ Botschaft enthalten. Außerdem sollte der Leser das Thema des Absatzes bereits in den ersten drei Wörtern, mindestens aber im ersten Satz erkennen.

\item Achte  auf eine logische Verbindung innerhalb und zwischen den Absätzen. Am einfachsten ist es, zuerst die Hauptgedanken aufzuschreiben und sie erst dann zu Absätzen auszubauen.

\end{itemize}

\subsection{Sprache}
\begin{itemize}
	
	\item Schreibe in kurzen und prägnanten Sätzen. 
	
\item Vermeide die Verwendung der ersten Person (``ich'' und ``wir'').
	
	\item Verwende keine Verkürzungen oder umgangssprachliche Formulierungen.

\end{itemize}

\subsection{Zitieren}
\begin{itemize}

\item Wenn du ein Dokument verfasst, das in irgendeiner Weise verbreitet wird (oder werden könnte), solltest du darauf achten, dass alle deine Abbildungen frei von Urheberrechtsproblemen sind. Das bedeutet: Entweder zeichnest du sie selbst (auch die Übernahme von Teilen aus anderen Bildern und deren Veränderung ist ein Problem) oder du holst eine Genehmigung ein und gibst in der Bildunterschrift an, dass du diese Genehmigung erhalten hast und von wem. Meistens sagen dir die Verlage, wie man das schreibt und haben sogar ein automatisiertes Verfahren, um die Erlaubnis zu erhalten, z. B. über RightsLink/ Copyright Clearance Center. Auch Bilder aus dem öffentlichen Bereich müssen ordnungsgemäß referenziert werden. Es gibt auch verschiedene Arten von Lizenzen, z. B. erlauben einige keine Änderungen an einem Bild.

\item Vermeide direkte Zitate (``...'') wann immer möglich. Formuliere stattdessen Aussagen um, wenn du zitierst.

\item Es ist eindeutig vorzuziehen, Primär- und Originalliteratur zu zitieren, nicht Rezensionen oder Lehrbücher. Gehe auf die Originalquellen zurück. Wenn du Rezensionen oder Bücher zitierst, tue  dies ausdrücklich, z. B. mit ``ein Überblick findet sich in...''.

\item Lese jede Arbeit die du zitierst so lange, bis du sie verstehst, oder zumindest den Teil der Arbeit, auf den du dich beziehst. Verlasse dich nicht nur auf Zusammenfassungen oder, schlimmer noch, auf Sekundärliteratur, die die Originalquelle (oft falsch) zitiert. Wenn du es für unmöglich hälst, die zitierten Quellen zu lesen und zu verstehen, schränke den Umfang deiner Arbeit so weit ein, bis es möglich ist.

\item Belege jede einzelne Behauptung, die keine offensichtliche Tatsache ist, entweder durch einen Verweis oder durch ein eigenes Experiment, eine Schlussfolgerung oder eine Berechnung. 

\end{itemize}

\subsection{Mathematik}
\begin{itemize}
	
	\item Definiere jede einzelne Variable explizit. Verlasse dich nicht auf ``selbsterklärende'' Indizes wie $x_{\mathrm{max}}$ oder auf den Erfahrungshintergrund des Lesers (z.B. um anzunehmen, dass $I$ ein elektrischer Strom sein muss). Wähle kurze Indizes, lieber einen als mehrere Buchstaben. 
	

\item Operatoren und gängige mathematische Funktionen werden im Hochformat gesetzt~\cite{RedBook2010}. Besonders häufig wird dies bei dem Differentialoperator oder bei sin und cos falsch gemacht. Beispiel: Die Ableitung der Variablen $x$ nach der Zeit $t$ ist
\begin{equation}
\dot{x}=\frac{\ud x}{\ud t} \, .\nonumber
\label{eq:derivative}
\end{equation}

\item Alles, was Text ist, sollte nicht als Variablen gesetzt werden. Dies gilt auch für Indizes von Variablen. Zum Beispiel, für eine Variable $\lambda$ mit dem beabsichtigten Index ``m'', wobei ``m'' eine Abkürzung für ``max'' ist, schreibe $\lambda_{\mathrm{m}}$ und nicht $\lambda_m$.

\item Es ist ein sehr häufiger Fehler, eckige Klammern um die Einheit zu setzen. Dies ist nach der SI- ~\cite{SIbrochure2006} und der ISO-Norm~\cite{ISOnormunits} nicht korrekt. So ist z.B. für eine Kraft $F$ die korrekte Verwendung des Operators $[F]=\si{N}$. Für Achsenbeschriftungen in Diagrammen ist die Division wie in ``$F/\si{N}$'' (bevorzugt) oder ``Kraft $F$ (in \si{N})'' zu verwenden. %ISO80000-1:2009 Quantities and units - Part 1: Allgemeines, verfügbar unter https://www.iso.org/obp/ui/#iso:std:iso:80000:-1:ed-1:v1:en.

\item Einheiten werden römisch (aufrecht) gesetzt, nicht kursiv (schräg).

\item Der Abstand zwischen einer Zahl und einer Einheit sollte etwas geringer sein als ein normaler Abstand. In \LaTeX verwende am besten {\tt \textbackslash ,} oder das Paket~{\tt units}, um sowohl den korrekten Abstand als auch den korrekten Satz zu gewährleisten. 
Beispiel: moment $M=\SI{3}\.{Nm}$.%Beachten Sie die Verwendung des "." in diesem Paket, um den entsprechenden Abstand zwischen zwei Einheiten zu erzeugen (zur Kennzeichnung der Multiplikation). Falls Sie sich fragen, warum dies notwendig ist, betrachten Sie den zweideutigen Fall von "3 ms". Heißt das "3 Millisekunden" oder "3 Meter mal Sekunden"? Alternativ kann man auch \cdot verwenden, um explizit zu multiplizieren.

%\item Beachte, dass auch Reglerverstärkungen Einheiten haben! Geben diese in der Arbeit an. Das Verständnis der Einheiten hilft dir auch bei der Wahl vernünftiger Verstärkungen während der Abstimmung.

\item Benutze  keinen Operator zwischen Variablen ($F=ma$), aber einen ``$\cdot$'' wenn nötig ($F=\SI{3}{kg}\cdot\SI{2}{m/s^2}$).

\item Beginne Sätze nicht mit Zahlen oder Variablennamen. Manchmal hilft es, Konstruktionen wie ``Der Parameter $k$'' oder ``Die Variable $x$'' oder ``Ein Wert von 20 ...'' zu verwenden.

\item Setze Skalare, Vektoren und Matrizen verschieden. Beispiel:
Die Massenmatrix $\M M_{\mathrm{r}}(\V{q})$ eines Robotermanipulators ist eine Funktion der verallgemeinerten Koordinaten des Roboters~$\V{q}$.
%Um dies zu aktivieren, verwenden Sie
%\newcommand{\V}[1]{\boldsymbol{#1}}%vectors
%\newcommand{\M}[1]{\mathbf{#1}}%matrices
%in der Präambel deines Dokuments.

\item Füge Formeln in Sätze ein. Setze dabei Punkte, wenn ein Satz nach Ihrer Gleichung endet, oder ein Komma, wenn es angebracht ist. 
\end{itemize}


\subsection{Abbildungen}

\begin{itemize}
\item Verweise auf alle Abbildungen und Tabellen im Text bevor sie erscheinen.

\item Verwende nach Möglichkeit Vektorgrafiken und raster den Text nicht. Du kannst Abbildungen als {\tt .eps} aus Matlab exportieren.

\item Vergewissere dich, dass deine Abbildungen im Schwarzweißdruck und von Farbenblinden perfekt lesbar sind. Verwende also nicht (nur) Farbcodes, sondern auch unterschiedliche Linienstile, Dicken oder Graustufen.

\item Der gesamte Text in den Abbildungen sollte ohne Zoom lesbar sein, auch für ältere Leser. Verwende daher keine Schriftgrößen, die kleiner als die Fußnotengröße sind. 

\item Verwende für Symbole in Abbildungen die gleiche Schriftart wie in Ihrem Haupttext. Ein Paket, das hier hilft, ist {\tt psfrag}. Es ersetzt während der Kompilierung den Text in deinen Abbildungen durch Text oder Variablen deiner Wahl.  
Beispiel: Eine Scheibe ändert ihre Winkelgeschwindigkeit $\omega$, wenn ein äußeres Drehmoment $\tau$ auf sie wirkt~(Abb.~\ref{fig:diskwithtorque}). %Verweisen Sie in Ihrem Text immer auf Abbildungen.
%Diese Abbildung zeigt, wie man den Befehl psfrag verwendet. 
\begin{figure}[htb]
	\psfrag{d}[l][l]{\sf disk} %sf bedeutet serifenlos, bevorzugter Stil in Illustrationen
	\psfrag{tauj}[c][c]{\sf $\tau$} %sf hat keine Auswirkung im mathematischen Modus
	\psfrag{phidj}[c][c]{\sf $\omega$}
	\psfrag{rj}[c][c]{\sf $r$}
	\centering
	\includegraphics[width=0.3\linewidth]{figures/Discwithtorque.eps}
	\caption[Kurztitel]{Ein Antriebsdrehmoment $\tau$ wirkt auf eine Scheibe mit dem Radius $r$ und bewirkt, dass die Scheibe ihre Winkelgeschwindigkeit $\omega$ ändert (Versuchen Sie, die Abbildung zu skalieren und beachten Sie die Schriftgröße).
	}
	\label{fig:diskwithtorque}
\end{figure}
\item Beim Zeichnen von Blockdiagrammen ist die IEC-Notation~\cite{IECControltech} zu verwenden, d.h. u.a. Kreise für Summenpunkte verwenden, Vorzeichen an der rechten Seite der eingehenden Aktionslinien positionieren, Punkte für Verzweigungspunkte verwenden, rechteckige Blöcke verwenden.

\end{itemize}

\subsection{Latex Manual}
Alles zum Thema Latex ist in \cite{Oettinger}


beschrieben.

\newpage
\newpage
\section*{HINWEISE ZUR GLIEDERUNG}